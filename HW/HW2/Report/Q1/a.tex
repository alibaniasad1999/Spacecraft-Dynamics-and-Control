\section{Question 1}
The space shuttle weighs approximately 12.5 tons, whose thrusters can simultaneously produce a total thrust of 53400 Newtons for orbital maneuvers. Assuming that the shuttle is initially in a 300 Km (altitude) circular Earth orbit, it is desired to use a single impulse to transfer the shuttle to a 250x300 Km elliptical orbit.
\subsection{part a}

$$
h = 
\sqrt{2\mu}\sqrt{\dfrac{r_a r_p}{r_a + r_p}}
$$

$$
v = \dfrac{h}{r}
$$

First orbit (circular):
$$
r = 6678
$$

For first circular orbit $r_a = r_p$.

$$
h = 51593 \to v = 7.7258_{km/\sec}
$$

Second orbit (elliptical):
$$
r_p = 6628, \quad r_a = 6678
$$

$$
h = 51496 \to v_a = 7.7113_{km/\sec}
$$

$$
\Delta v = v_a - v = 0.0145_{km/\sec}
$$